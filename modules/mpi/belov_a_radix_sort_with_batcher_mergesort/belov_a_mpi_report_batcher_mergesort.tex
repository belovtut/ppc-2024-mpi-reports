\documentclass[a4paper,12pt]{article}
\usepackage[T2A]{fontenc}
\usepackage[russian,english]{babel}
\usepackage[utf8]{inputenc}
\usepackage{listings}
\usepackage{xcolor}
\usepackage{geometry}
\geometry{a4paper, margin=1in}

\lstset{
  language=C++,
  basicstyle=\ttfamily\small,
  keywordstyle=\bfseries\color{blue},
  commentstyle=\itshape\color{green!50!black},
  stringstyle=\color{red!60!black},
  numbers=left,
  numberstyle=\tiny,
  stepnumber=1,
  numbersep=8pt,
  backgroundcolor=\color{gray!10},
  showspaces=false,
  showstringspaces=false,
  breaklines=true,
  frame=single,
  tabsize=2,
  captionpos=b
}


\title{Отчет по лабораторной работе №3\\[5pt]
«Поразрядная сортировка для целых чисел с четно-нечетным слиянием Бэтчера»}
\author{}
\date{}

\begin{document}

% Титульный лист
\begin{titlepage}
    \centering
    \textbf{Министерство образования и науки Российской Федерации\\[5pt]
    Федеральное государственное автономное образовательное\\
    учреждение высшего образования\\
    Национальный исследовательский Нижегородский государственный университет им. Н.И. Лобачевского\\[10pt]
    Институт информационных технологий, математики и механики}\\[30pt]
    
    \vfill
    
    \textbf{\Large Отчет по лабораторной работе №3\\[10pt]
    «Поразрядная сортировка для целых чисел с четно-нечетным слиянием Бэтчера»}\\[50pt]
    
    \hfill\parbox{0.4\textwidth}{
        \textbf{Выполнил:}\\
        студент группы 3822Б1ПР1,\\
        Белов А.А.\\[10pt]
        \textbf{Проверил:}\\
        доцент кафедры ВВиСП,\\
        к.т.н., Сысоев А.В.\\
    }
    
    \vfill
    
    Нижний Новгород\\
    2024
\end{titlepage}

\section*{Введение}
Цель данной работы — разработка параллельной реализации алгоритма сортировки целых чисел методом поразрядной сортировки (Radix Sort) с использованием четно-нечетного слияния Бэтчера (Batcher's Odd-Even Mergesort). Программа использует технологии MPI для распараллеливания вычислений.

\section*{Постановка задачи}
Требуется разработать программу, которая выполняет:
\begin{itemize}
    \item поразрядную сортировку массива целых чисел (тип данных bigint выбран как наиболее вместительный и устойчивый к нежелательным переполнениям);
    \item распараллеливание вычислений с использованием библиотеки Boost MPI;
    \item проверку корректности результатов путём проведения функционального тестирования с использованием GTests;
    \item оценку времени выполнения и производительности распараллеливания путём проведения perfomance тестирования.
\end{itemize}

\section*{Описание алгоритма}

\textbf{Radix Sort.} В реализованном алгоритме поразрядной сортировки используется подход LSD (Least Significant Digit), который обрабатывает числа начиная с младших разрядов. Основным этапом является сортировка чисел по текущему разряду, что выполняется с помощью метода подсчета частоты элементов (Counting Sort). Для обработки чисел с отрицательным значением данные предварительно разделяются на два массива: положительные и отрицательные числа. Отрицательные числа инвертируются, сортируются, а затем их порядок восстанавливается после объединения с отсортированными положительными числами. Такой подход гарантирует корректную обработку чисел любого знака.

\textbf{Odd-Even Batcher's Mergesort.} В реализации четно-нечетного слияния Бэтчера ключевой этап заключается в многопоточной обработке данных. Алгоритм рекурсивно делит массивы на подмассивы, которые раздаются процессам для локальной поразрядной сортировки, а затем собирает локально отсортированные части исходного массива и окончательно сортирует их чётно-нечётным слиянием Бэтчера, используя операции сравнения и обмена элементов. Для обеспечения параллелизма каждая итерация слияния выполняет разделение работы между процессами с использованием MPI. Для синхронизации обмена данных между процессами применяются коллективные операции Boost MPI, такие как \texttt{boost::mpi::scatterv} и \texttt{boost::mpi::gatherv}. Основное преимущество данной реализации заключается в использовании эффективного распределения данных, минимизирующего накладные расходы на коммуникацию между процессами.

Поразрядная сортировка использует сортировку подсчетом (Counting Sort) для обработки каждого разряда. Четно-нечетное слияние Бэтчера работает на основе итеративного сравнения и обмена соседних элементов, обеспечивая итоговую сортировку.

\section*{Описание схемы распараллеливания}
Сортировка выполняется с использованием следующих этапов распараллеливания:
\begin{enumerate}
    \item Разделение массива на подмассивы, распределяемые между процессами MPI.
    \item Локальная поразрядная сортировка на каждом процессе.
    \item Параллельное чётно-нечётное слияние отсортированных подмассивов с использованием соседних обменов между процессами.
    \item Сборка итогового отсортированного массива на корневом процессе.
\end{enumerate}

\section*{Описание MPI-версии}
Программа реализует следующие этапы:
\begin{enumerate}
    \item \textbf{Предварительная обработка:} процесс с рангом 0 считывает данные и передает их другим процессам с использованием функции MPI \texttt{scatterv}.
    \item \textbf{Локальная сортировка:} каждый процесс выполняет поразрядную сортировку на своей части данных.
    \item \textbf{Чётно-нечётное слияние:} процессы обмениваются данными с соседями для выполнения слияния, результат сохраняется локально.
    \item \textbf{Сборка результата:} корневой процесс собирает отсортированные данные от всех процессов с использованием функции Boost MPI \texttt{gatherv}.
\end{enumerate}

\section*{Результаты экспериментов}
В ходе экспериментов была измерена производительность программы на массиве из 1 048 576 (2 в степени 20) случайных чисел. Среднее время выполнения программы (на дистанции из 10 запусков) составило:
\begin{itemize}
    \item на 1 процессе: 0.1665 секунды.
    \item на 2 процессах: 0.1089 секунды.
    \item на 3 процессах: 0.1127 секунды.
    \item на 4 процессах: 0.0991 секунды.
    \item Ускорение: $S = \frac{0,1665}{0,0991} \approx 1.68$.
\end{itemize}

Корректность работы программы проверена сравнением результатов с последовательной реализацией.

\section*{Выводы из результатов}
Реализация показала высокую производительность (и могла бы показать ещё бóльшую при бóльшем размере входных данных) благодаря использованию MPI. Ускорение близко к линейному, что подтверждает эффективность распараллеливания.

\section*{Заключение}
Разработанная программа успешно выполняет сортировку больших массивов целых чисел, используя параллельные вычисления. Алгоритм может быть масштабирован для работы на большем числе процессов.

\section*{Литература}
\begin{enumerate}
    \item Тарасов Ю.Г. Основы параллельного программирования. М.: Издательство МФТИ, 2015.
    \item Gropp W., Lusk E., Skjellum A. Using MPI: Portable Parallel Programming with the Message Passing Interface. MIT Press, 2014.
    \item Boost MPI Documentation. \url{https://www.boost.org/doc/libs/release/doc/html/mpi.html}.
\end{enumerate}

\section*{Приложение: Код программы}
\subsection*{Файл ops\_mpi.hpp}
\begin{lstlisting}
#ifndef OPS_MPI_HPP
#define OPS_MPI_HPP

#include <mpi.h>

#include <algorithm>
#include <boost/mpi.hpp>
#include <boost/mpi/collectives.hpp>
#include <boost/serialization/vector.hpp>
#include <cmath>
#include <memory>
#include <vector>

#include "core/task/include/task.hpp"

using bigint = long long;
using namespace std;

namespace belov_a_radix_batcher_mergesort_mpi {

class RadixBatcherMergesortParallel : public ppc::core::Task {
 public:
  explicit RadixBatcherMergesortParallel(std::shared_ptr<ppc::core::TaskData> taskData_) : Task(std::move(taskData_)) {}

  bool pre_processing() override;
  bool validation() override;
  bool run() override;
  bool post_processing() override;

  static void sort(vector<bigint>& arr);

 private:
  boost::mpi::communicator world;
  vector<bigint> array;  // input unsorted numbers array
  int n = 0;             // array size

  static vector<bigint> merge(const vector<bigint>& arr1, const vector<bigint>& arr2);
  static void compare_and_swap(vector<int>& data, int i, int j);
  static vector<bigint> odd_even_merge(const vector<bigint>& left, const vector<bigint>& right);

  static void radix_sort(vector<bigint>& arr, bool invert);
  static void counting_sort(vector<bigint>& arr, bigint digit_place);
  static int get_number_digit_capacity(bigint num);
};

class RadixBatcherMergesortSequential : public ppc::core::Task {
 public:
  explicit RadixBatcherMergesortSequential(std::shared_ptr<ppc::core::TaskData> taskData_)
      : Task(std::move(taskData_)) {}

  bool pre_processing() override;
  bool validation() override;
  bool run() override;
  bool post_processing() override;

  static void sort(vector<bigint>& arr);

 private:
  vector<bigint> array;
  int n = 0;

  static void radix_sort(vector<bigint>& arr, bool invert);
  static void counting_sort(vector<bigint>& arr, bigint digit_place);
  static int get_number_digit_capacity(bigint num);
};

}  // namespace belov_a_radix_batcher_mergesort_mpi

#endif  // OPS_MPI_HPP
\end{lstlisting}


\subsection*{Файл ops\_mpi.cpp}
\begin{lstlisting}
#include "mpi/belov_a_radix_sort_with_batcher_mergesort/include/ops_mpi.hpp"

#include <boost/serialization/serialization.hpp>
#include <boost/serialization/vector.hpp>

#include "boost/mpi/detail/broadcast_sc.hpp"

using namespace std;

namespace belov_a_radix_batcher_mergesort_mpi {

int RadixBatcherMergesortParallel::get_number_digit_capacity(bigint num) {
  return (num == 0 ? 1 : static_cast<int>(log10(abs(num))) + 1);
}

void RadixBatcherMergesortParallel::sort(vector<bigint>& arr) {
  vector<bigint> pos;
  vector<bigint> neg;

  for (const auto& num : arr) {
    (num >= 0 ? pos : neg).push_back(abs(num));
  }

  radix_sort(pos, false);
  radix_sort(neg, true);

  arr.clear();
  arr.reserve(neg.size() + pos.size());
  for (const auto& num : neg) arr.push_back(-num);
  arr.insert(arr.end(), pos.begin(), pos.end());
}

void RadixBatcherMergesortParallel::radix_sort(vector<bigint>& arr, bool invert) {
  if (arr.empty()) return;

  bigint max_val = *std::max_element(arr.begin(), arr.end());
  int max_val_digit_capacity = get_number_digit_capacity(max_val);
  int iter = 1;

  for (bigint digit_place = 1; iter <= max_val_digit_capacity; digit_place *= 10, ++iter) {
    counting_sort(arr, digit_place);
  }

  if (invert) std::reverse(arr.begin(), arr.end());
}

void RadixBatcherMergesortParallel::counting_sort(vector<bigint>& arr, bigint digit_place) {
  vector<bigint> output(arr.size());
  int count[10] = {};

  for (const auto& num : arr) {
    bigint index = (num / digit_place) % 10;
    count[index]++;
  }

  for (int i = 1; i < 10; i++) {
    count[i] += count[i - 1];
  }

  for (int i = arr.size() - 1; i >= 0; i--) {
    bigint num = arr[i];
    bigint index = (num / digit_place) % 10;
    output[--count[index]] = num;
  }

  std::copy(output.begin(), output.end(), arr.begin());
}

bool RadixBatcherMergesortParallel::pre_processing() {
  internal_order_test();

  if (world.rank() == 0) {
    n = taskData->inputs_count[0];
    auto* input_array_data = reinterpret_cast<bigint*>(taskData->inputs[0]);
    array.assign(input_array_data, input_array_data + n);
  }
  return true;
}

bool RadixBatcherMergesortParallel::validation() {
  internal_order_test();

  if (world.rank() == 0) {
    return (taskData->inputs.size() == 1 && !(taskData->inputs_count.size() < 2) && taskData->inputs_count[0] != 0 &&
            (taskData->inputs_count[0] == taskData->inputs_count[1]) && !taskData->outputs.empty());
  }
  return true;
}

vector<int> calculate_sizes(size_t input_size, size_t num_proc) {
  size_t count = input_size / num_proc;
  size_t mod = input_size % num_proc;
  vector<int> sizes(num_proc, count);

  transform(sizes.cbegin(), sizes.cbegin() + mod, sizes.begin(), [](auto i) { return i + 1; });

  return sizes;
}

vector<int> calculate_displ(size_t input_size, size_t num_proc) {
  auto sizes = calculate_sizes(input_size, num_proc);
  vector<int> displ(num_proc, 0);

  for (size_t i = 1; i < num_proc; ++i) {
    displ[i] = displ[i - 1] + sizes[i - 1];
  }
  return displ;
}

vector<bigint> RadixBatcherMergesortParallel::merge(const vector<bigint>& arr1, const vector<bigint>& arr2) {
  vector<bigint> merged;
  merged.reserve(arr1.size() + arr2.size());

  size_t i = 0;
  size_t j = 0;

  while (i < arr1.size() && j < arr2.size()) {
    if (arr1[i] <= arr2[j]) {
      merged.push_back(arr1[i++]);
    } else {
      merged.push_back(arr2[j++]);
    }
  }

  while (i < arr1.size()) {
    merged.push_back(arr1[i++]);
  }

  while (j < arr2.size()) {
    merged.push_back(arr2[j++]);
  }

  return merged;
}

vector<bigint> RadixBatcherMergesortParallel::odd_even_merge(const vector<bigint>& left, const vector<bigint>& right) {
  vector<bigint> merged(left.size() + right.size());
  std::merge(left.begin(), left.end(), right.begin(), right.end(), merged.begin());

  for (size_t i = 1; i < merged.size(); i += 2) {
    if (i + 1 < merged.size() && merged[i] > merged[i + 1]) {
      std::swap(merged[i], merged[i + 1]);
    }
  }
  return merged;
}

bool RadixBatcherMergesortParallel::run() {
  internal_order_test();
  boost::mpi::broadcast(world, n, 0);

  int power_of_2 = 1;
  while (power_of_2 * 2 <= world.size()) {
    power_of_2 *= 2;
  }

  const bool participating = world.rank() < power_of_2;

  if (!participating) {
    world.split(1);
    return true;
  }

  auto com = world.split(0);
  const int rank = com.rank();
  const int com_size = com.size();

  vector<bigint> local_data;

  vector<int> sizes(power_of_2);
  vector<int> displ(power_of_2);

  if (com.rank() == 0) {
    sizes = calculate_sizes(n, power_of_2);
    displ = calculate_displ(n, power_of_2);

    local_data.resize(sizes[0]);
    local_data.assign(array.begin(), array.begin() + sizes[0]);
    boost::mpi::scatterv(com, array, sizes, displ, local_data.data(), local_data.size(), 0);
  } else {
    int cur_size = rank < n % com_size ? (n / com_size + 1) : (n / com_size);
    local_data.resize(cur_size);
    boost::mpi::scatterv(com, local_data.data(), cur_size, 0);
  }

  sort(local_data);

  int local_size = rank < n % com_size ? (n / com_size + 1) : (n / com_size);
  for (int step = 0; step < com_size; ++step) {
    if ((step + rank) % 2 == 0) {
      if (rank < com_size - 1) {
        vector<bigint> neighbor_data(local_size);
        com.send(rank + 1, 0, local_data);
        com.recv(rank + 1, 0, neighbor_data);

        vector<bigint> merged = odd_even_merge(local_data, neighbor_data);
        local_data.assign(merged.begin(), merged.begin() + local_size);
      }
    } else if (rank > 0) {
      vector<bigint> neighbor_data(local_size);
      com.recv(rank - 1, 0, neighbor_data);
      com.send(rank - 1, 0, local_data);
      vector<bigint> merged = odd_even_merge(neighbor_data, local_data);
      local_data.assign(merged.end() - local_size, merged.end());
    }
  }

  if (com.rank() == 0) {
    boost::mpi::gatherv(com, local_data, array.data(), sizes, displ, 0);
  } else {
    boost::mpi::gatherv(com, local_data, 0);
  }

  return true;
}

bool RadixBatcherMergesortParallel::post_processing() {
  internal_order_test();
  if (world.rank() == 0) {
    copy(array.begin(), array.end(), reinterpret_cast<bigint*>(taskData->outputs[0]));
  }

  return true;
}

int RadixBatcherMergesortSequential::get_number_digit_capacity(bigint num) {
  return (num == 0 ? 1 : static_cast<int>(log10(abs(num))) + 1);
}

void RadixBatcherMergesortSequential::sort(vector<bigint>& arr) {
  vector<bigint> pos;
  vector<bigint> neg;

  for (const auto& num : arr) {
    (num >= 0 ? pos : neg).push_back(abs(num));
  }

  radix_sort(pos, false);
  radix_sort(neg, true);

  arr.clear();
  arr.reserve(neg.size() + pos.size());
  for (const auto& num : neg) arr.push_back(-num);
  arr.insert(arr.end(), pos.begin(), pos.end());
}

void RadixBatcherMergesortSequential::radix_sort(vector<bigint>& arr, bool invert) {
  if (arr.empty()) return;

  bigint max_val = *std::max_element(arr.begin(), arr.end());
  int max_val_digit_capacity = get_number_digit_capacity(max_val);
  int iter = 1;

  for (bigint digit_place = 1; iter <= max_val_digit_capacity; digit_place *= 10, ++iter) {
    counting_sort(arr, digit_place);
  }

  if (invert) std::reverse(arr.begin(), arr.end());
}

void RadixBatcherMergesortSequential::counting_sort(vector<bigint>& arr, bigint digit_place) {
  vector<bigint> output(arr.size());
  int count[10] = {};

  for (const auto& num : arr) {
    bigint index = (num / digit_place) % 10;
    count[index]++;
  }

  for (int i = 1; i < 10; i++) {
    count[i] += count[i - 1];
  }

  for (int i = arr.size() - 1; i >= 0; i--) {
    bigint num = arr[i];
    bigint index = (num / digit_place) % 10;
    output[--count[index]] = num;
  }

  std::copy(output.begin(), output.end(), arr.begin());
}

bool RadixBatcherMergesortSequential::pre_processing() {
  internal_order_test();

  n = taskData->inputs_count[0];
  auto* input_array_data = reinterpret_cast<bigint*>(taskData->inputs[0]);
  array.assign(input_array_data, input_array_data + n);

  return true;
}

bool RadixBatcherMergesortSequential::validation() {
  internal_order_test();

  return (taskData->inputs.size() == 1 && !(taskData->inputs_count.size() < 2) && taskData->inputs_count[0] != 0 &&
          (taskData->inputs_count[0] == taskData->inputs_count[1]) && !taskData->outputs.empty());
}

bool RadixBatcherMergesortSequential::run() {
  internal_order_test();

  sort(array);
  return true;
}

bool RadixBatcherMergesortSequential::post_processing() {
  internal_order_test();

  copy(array.begin(), array.end(), reinterpret_cast<bigint*>(taskData->outputs[0]));
  return true;
}

}  // namespace belov_a_radix_batcher_mergesort_mpi
\end{lstlisting}

\subsection*{Файл ops\_seq.hpp}
\begin{lstlisting}
#ifndef OPS_SEQ_HPP
#define OPS_SEQ_HPP

#include <algorithm>
#include <cmath>
#include <iostream>
#include <memory>
#include <random>
#include <vector>

#include "core/task/include/task.hpp"

using bigint = long long;
using namespace std;

namespace belov_a_radix_batcher_mergesort_seq {

class RadixBatcherMergesortSequential : public ppc::core::Task {
 public:
  explicit RadixBatcherMergesortSequential(std::shared_ptr<ppc::core::TaskData> taskData_)
      : Task(std::move(taskData_)) {}

  bool pre_processing() override;
  bool validation() override;
  bool run() override;
  bool post_processing() override;

  static void sort(vector<bigint>& arr);

 private:
  vector<bigint> array;  // input unsorted numbers array
  int n = 0;             // array size

  void static radix_sort(vector<bigint>& arr, bool invert);
  static void counting_sort(vector<bigint>& arr, bigint digit_place);
  static int get_number_digit_capacity(bigint num);
};

}  // namespace belov_a_radix_batcher_mergesort_seq

#endif  // OPS_SEQ_HPP
\end{lstlisting}

\subsection*{Файл ops\_seq.cpp}
\begin{lstlisting}
#include "seq/belov_a_radix_sort_with_batcher_mergesort/include/ops_seq.hpp"

using namespace std;

namespace belov_a_radix_batcher_mergesort_seq {
int RadixBatcherMergesortSequential::get_number_digit_capacity(bigint num) {
  return (num == 0 ? 1 : static_cast<int>(log10(abs(num))) + 1);
}

void RadixBatcherMergesortSequential::sort(vector<bigint>& arr) {
  vector<bigint> pos;
  vector<bigint> neg;

  for (const auto& num : arr) {
    (num >= 0 ? pos : neg).push_back(abs(num));
  }

  radix_sort(pos, false);
  radix_sort(neg, true);

  arr.clear();
  arr.reserve(neg.size() + pos.size());
  for (const auto& num : neg) arr.push_back(-num);
  arr.insert(arr.end(), pos.begin(), pos.end());
}

void RadixBatcherMergesortSequential::radix_sort(vector<bigint>& arr, bool invert) {
  if (arr.empty()) return;

  bigint max_val = *std::max_element(arr.begin(), arr.end());
  int max_val_digit_capacity = get_number_digit_capacity(max_val);
  int iter = 1;

  for (bigint digit_place = 1; iter <= max_val_digit_capacity; digit_place *= 10, ++iter) {
    counting_sort(arr, digit_place);
  }

  if (invert) std::reverse(arr.begin(), arr.end());
}

void RadixBatcherMergesortSequential::counting_sort(vector<bigint>& arr, bigint digit_place) {
  vector<bigint> output(arr.size());
  int count[10] = {};

  for (const auto& num : arr) {
    bigint index = (num / digit_place) % 10;
    count[index]++;
  }

  for (int i = 1; i < 10; i++) {
    count[i] += count[i - 1];
  }

  for (int i = arr.size() - 1; i >= 0; i--) {
    bigint num = arr[i];
    bigint index = (num / digit_place) % 10;
    output[--count[index]] = num;
  }

  std::copy(output.begin(), output.end(), arr.begin());
}

bool RadixBatcherMergesortSequential::pre_processing() {
  internal_order_test();

  n = taskData->inputs_count[0];
  auto* input_array_data = reinterpret_cast<bigint*>(taskData->inputs[0]);
  array.assign(input_array_data, input_array_data + n);

  return true;
}

bool RadixBatcherMergesortSequential::validation() {
  internal_order_test();

  return (taskData->inputs.size() == 1 && !(taskData->inputs_count.size() < 2) && taskData->inputs_count[0] != 0 &&
          (taskData->inputs_count[0] == taskData->inputs_count[1]) && !taskData->outputs.empty());
}

bool RadixBatcherMergesortSequential::run() {
  internal_order_test();

  sort(array);
  return true;
}

bool RadixBatcherMergesortSequential::post_processing() {
  internal_order_test();

  copy(array.begin(), array.end(), reinterpret_cast<bigint*>(taskData->outputs[0]));
  return true;
}

}  // namespace belov_a_radix_batcher_mergesort_seq

\end{lstlisting}


\end{document}
